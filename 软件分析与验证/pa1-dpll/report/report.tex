\documentclass[12pt, UTF8]{article}
\usepackage[a4paper, scale = 0.8]{geometry}
\usepackage{ctex}

\usepackage{listings}
\usepackage{xcolor}
\usepackage{color}
\definecolor{GrayCodeBlock}{RGB}{241, 241, 241}
\definecolor{BlackText}{RGB}{110, 107, 94}
\definecolor{RedTypename}{RGB}{182, 86, 17}
\definecolor{GreenString}{RGB}{96, 172, 57}
\definecolor{PurpleKeyword}{RGB}{184, 84, 212}
\definecolor{GrayComment}{RGB}{170, 170, 170}
\definecolor{GoldDocumentation}{RGB}{180, 165, 45}
\lstset {
  columns = fullflexible, keepspaces = true, showstringspaces=false, breaklines = true, frame = single, framesep = 0pt, framerule = 0pt, framexleftmargin = 4pt, framexrightmargin = 4pt, framextopmargin = 5pt, framexbottommargin = 3pt, xleftmargin = 4pt, xrightmargin = 4pt,
  backgroundcolor = \color{GrayCodeBlock},
  basicstyle = \ttfamily\color{BlackText},
  keywordstyle = \color{PurpleKeyword},
  ndkeywordstyle = \color{RedTypename},
  stringstyle = \color{GreenString},
  commentstyle = \color{GrayComment}
}

\usepackage{graphicx}
\usepackage{amsmath}

\usepackage[colorlinks, linkcolor = red, anchorcolor = blue, citecolor = green]{hyperref}

\renewcommand\thesection{\arabic{section}}

\title{软件分析与验证PA1报告}
\author{李晨昊 2017011466}
\begin{document}
\maketitle
\tableofcontents

\section{简介}

我删除了\lstinline|DPLL.cpp|文件(同时也在\lstinline|CMakeList.txt|中去掉了它),而把逻辑都是现在了\lstinline|DPLL.h|文件中,这主要是个人习惯因素。

\lstinline|DPLL.h.basic|文件中包含了无backjump版本的DPLL求解器,把现在的\lstinline|DPLL.h|替换成它即可运行无backjump版本的求解器。

\lstinline|bitset.hpp|文件中实现了一个简单的\lstinline|bitset|,backjump版本的求解器中用到了它。

\section{基础部分实现思路}

\subsection{数据结构}

我定义了如下基础数据结构:

\begin{enumerate}
  \item
    \begin{lstlisting}[language = C++]
struct Var {
  u32 id : 31;
  u32 flag : 1;
};
    \end{lstlisting}

    \lstinline|Var|逻辑上表示一个整数和布尔值的二元组,只是用bitfield来节约了一点空间。它在不同的环境中有不同的含义,例如它可以表示子句中的一个文字,用\lstinline|id|表示变量编号,\lstinline|flag|表示是否有否定(为了方便,我没有使用dimacs中的带符号整数来表示字面值)。
  \item
    \begin{lstlisting}[language = C++]
struct VarInfo {
  enum {
    Undef = 0, Pos = 1, Neg = -1
  } state;
  std::vector<u32> pos_clauses;
  std::vector<u32> neg_clauses;
};
    \end{lstlisting}

    \lstinline|VarInfo|记录一个变量的状态,\lstinline|state|表示当前的部分解释中该变量的值,\lstinline|pos_clauses|和\lstinline|neg_clauses|存储这个变量在哪些子句中出现了,分别表示以正文字和负文字形式出现。它们用来协助进行unit propagation。
\end{enumerate}

\subsection{预处理}

预处理部分主要做以下几个操作:

\begin{enumerate}
  \item
    将带符号整数表示的文字转化成\lstinline|Var|表示的文字。
  \item
    做一个简单的优化:处理所有由单个文字组成的子句,为了让整个式子成立,子句的值必须为真,所以可以立即确定这个文字的变量的值。对于其它包含这个变量的子句,如果对应文字为真则删除这个子句,如果为假则删除这个文字。这个操作需要反复进行,直到不存在单个文字组成的子句。
    
    这个操作的主要目的其实不是优化,而是为了方便后续处理,后面会提到。
  \item
    维护每个变量的\lstinline|pos_clauses|和\lstinline|neg_clauses|。
\end{enumerate}

\subsection{核心算法}

大致的伪代码如下:

\begin{lstlisting}
while exist undefined variable x
  seq = []
  stk = [(x, true)]
  is_decision = true
  while stk is not empty
    (var, pos) = stk.pop()
    if var is defined
      if the old definition is conflict with pos
        if exist last definition d
          remove seq[index of d : end of seq]
          stk = [(var(d), ~pos(d))] 
          continue
        else
          return false
    else
      seq += (x, is_decision)
      is_decision = false
      vars[x] = pos
      unit prop on x, push newly inferred var to stk
return true
\end{lstlisting}

在进行unit prop的过程中维护一个栈,确保能够一次性找到所有传播的变量。

算法不会主动检测当前解释是否满足模型,判断冲突的唯一方法是:一个用unit prop得到的变量值与记录在\lstinline|vars|中的值冲突。这样之所以成立,是因为预处理阶段保证了不存在只包含一个文字的子句,假设当前解释不满足模型,那么必然存在一个子句中所有变量都有定义,但当倒数第二个变量被定义的时候,必然可以用unit prop确定最后一个变量的值,从而发现冲突。

\section{backjump部分实现思路}

对于一个冲突$P$和$\lnot P$我选择的冲突节点是:推导出$P$和$\lnot P$的所有决策变量的并集。为了得到这个信息,(逻辑上)为每个变量维护一个集合$decisions$,在上面的伪代码中,在\lstinline|vars|中记录值的同时维护集合,对于决策变量$x$,值为$\{x\}$;对于由子句$cls$进行unit prop得到的变量$x$,值为$\bigcup_{lit \in cls, var(lit) \ne x} decisions(var(lit))$。

实际实现的时候,用一个长度为$\left | Vars \right |$的\lstinline|bitset|来表示集合。为了更进一步优化性能,实现中不是为每个变量申请\lstinline|bitset|的空间,而是为所有变量申请一块空间,用变量的偏移量在其中访问。

backjump与基础版本中的backtrack在同一实际被调用,大致的伪代码为:

\begin{lstlisting}
if exist last definition d
  assert d is in conflict node set
  last = index of the first decision variable in seq, that is
    - before d (inclusive)
    - after the first variable before d & in conflict node set (exlusive)
  construct new_cls from conflict node set and old state
  update pos_clauses & neg_clauses according to new_cls
  add new_cls to clauses
  reset the states of vars in seq[last : end of seq]
  remove seq[last : end of seq]
  stk = [(var(d), ~pos(d))] 
  continue
else
  return false
\end{lstlisting}

\section{性能比较}

我在网络上寻找了一些测例,从中选出了无backjump版本和包含backjump版本的性能表现差距比较大的几个:\newline

\begin{center}
  \begin{tabular}{|c|c|c|}
    \hline            & backtrack & backjump \\
    \hline uf100-0250 & 745.393ms & 1.34635ms \\
    \hline uf100-0842 & 128.788ms & 0.329696ms \\
    \hline uf100-0469 & 298.535ms & 0.797293ms \\
    \hline
  \end{tabular}
\end{center}

不过仍然需要指出的是,对于许多测例backjump版本没有特别明显的性能优势,甚至在常数上稍劣一些,这可能与我选择冲突节点的策略有关,如果采用更复杂的策略也许会有更好的效果。

\section{参考资料}

\href{https://www.cs.ubc.ca/~hoos/SATLIB/benchm.html}{https://www.cs.ubc.ca/~hoos/SATLIB/benchm.html}

\end{document}

