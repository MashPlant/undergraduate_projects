\documentclass[12pt, UTF8]{article}
\usepackage[a4paper, scale = 0.8]{geometry}
\usepackage{ctex}

\usepackage{listings}
\usepackage{xcolor}
\usepackage{color}
\usepackage{diagbox}
\definecolor{GrayCodeBlock}{RGB}{241, 241, 241}
\definecolor{BlackText}{RGB}{0, 0, 0}
\definecolor{RedTypename}{RGB}{182, 86, 17}
\definecolor{GreenString}{RGB}{96, 172, 57}
\definecolor{PurpleKeyword}{RGB}{184, 84, 212}
\definecolor{GrayComment}{RGB}{170, 170, 170}
\definecolor{GoldDocumentation}{RGB}{180, 165, 45}
\lstset {
  columns = fullflexible, keepspaces = true, showstringspaces=false, breaklines = true, frame = single, framesep = 0pt, framerule = 0pt, framexleftmargin = 4pt, framexrightmargin = 4pt, framextopmargin = 5pt, framexbottommargin = 3pt, xleftmargin = 4pt, xrightmargin = 4pt,
  backgroundcolor = \color{GrayCodeBlock},
  basicstyle = \ttfamily\color{BlackText},
  keywordstyle = \color{PurpleKeyword},
  ndkeywordstyle = \color{RedTypename},
  stringstyle = \color{GreenString},
  commentstyle = \color{GrayComment}
}

\usepackage{graphicx}
\usepackage{amsmath}

\usepackage[colorlinks, linkcolor = red, anchorcolor = blue, citecolor = green]{hyperref}

\renewcommand\thesection{\arabic{section}}

\title{KV存储引擎实验报告}
\author{李晨昊 2017011466}
\begin{document}
\maketitle
\tableofcontents

\section{简介}

实现一个简单的KV存储引擎,支持写入键值对,搜索键和的范围搜索功能,并且保证了写入的崩溃一致性和证线性一致性。

我把几乎全部的代码都实现在了\lstinline|engine_race.h|中,这是我的个人习惯,没有什么特殊的意义。

我修改了\lstinline|Makefile|,将\lstinline|-std=c++11|修改成了\lstinline|-std=c++17|,因为我用到了其中的一些新特性。希望助教评测时用的编译器足够新。

\section{实现思路}

\subsection{基本数据结构}

我使用内存中的\lstinline|std::map|来存储和检索键值对,持久存储的部分则没有复杂的数据结构,可以简单理解成记录下来了所有插入的键值对的序列。持久存储的每一项逻辑上可以看成这样的数据结构:

\begin{lstlisting}[language = C++, morekeywords = { u8, u32, alignas }]
struct alignas(4096) Data {
  u32 key_sz;
  u8 key[key_sz];
  u32 val_sz;
  u8 val[val_sz];
};
\end{lstlisting}

读取和写入硬盘的时候都是按照这样的格式。

\subsection{写入的线性一致性}

也就是要保证存储引擎的线程安全。我使用\lstinline|pthread_rwlock_t|来实现写入的线性一致性,在\lstinline|Read|操作的时候使用读锁,在\lstinline|Write|操作的时候使用写锁。具体实现上我还做了一些优化,包括:

\begin{itemize}
  \item 尽可能最小化临界区。\lstinline|Write|操作的时候只保护了对内存中的数据结构的修改,没有保护\lstinline|write|调用,这是因为\lstinline|write|本身保证原子性,即只要写入了一定数目的内容,则保证这些内容不会和别的\lstinline|write|写入的内容交叉。不过如果\lstinline|write|没能一次性写入所有的内容,则这样的原子性对我们的应用就没有意义了,不应该继续调用\lstinline|write|来写入后续的内容了,我选择直接返回\lstinline|kIncomplete|。
  \item 将\lstinline|key|的第一个字节用作它的散列值,把不同的键分散到256个槽中。这个散列函数当然不算一个好的散列函数,但是还是能有效减少冲突,而且保持了字符串间的顺序,不会影响\lstinline|Range|的实现。
  
  在\lstinline|Range|操作的时候,先锁住所有相关的槽,在完成访问后再解锁它们。由于这里锁的获取是有序的,根据死锁产生的必要条件,可以保证不会产生死锁。
\end{itemize}

\subsection{写入的崩溃一致性}

我使用Linux的Direct IO来实现写入的崩溃一致性:

\begin{lstlisting}[language = C++, ndkeywords = {open}]
f.fd = open(buf.c_str(), O_RDWR | O_CREAT | O_APPEND | O_DIRECT, 0666);
\end{lstlisting}

后续直接对\lstinline|f.fd|进行IO操作即可。不过需要注意的是Direct IO模式下要求每次写入的长度都按页对齐,所以写入键值对的时候,需要先把必要的信息写入一个按页对齐的缓冲区,再把缓冲区中的内容写入文件:

\begin{lstlisting}[language = C++, morekeywords = { u8, u32, usize }, ndkeywords = { memcpy, write }]
u32 tot_sz = (8 + key_sz + val_sz + PAGE_SIZE_M1) & ~PAGE_SIZE_M1;
u8 *buf = (u8 *)((usize)(BUF + PAGE_SIZE_M1) & ~PAGE_SIZE_M1);
*(u32 *)buf = key_sz;
memcpy(buf + 4, key.data(), key_sz);
*(u32 *)(buf + 4 + key_sz) = val_sz;
memcpy(buf + 8 + key_sz, val.data(), val_sz);
if (write(f.fd, buf, tot_sz) != tot_sz) {
  // if IO cannot be performed in one `write` call, there will be synchronization problems
  return kIncomplete;
}
\end{lstlisting}

这样每次最少写入一个页的内容。这可能会导致一定的空间浪费,不过也没有什么更简便的做法了。

这个缓冲区\lstinline|BUF|是一个\lstinline|thread_local|变量,所以对它的操作不需要锁的保护。我尝试过用\lstinline|alignas|关键字来标记\lstinline|BUF|,让它按页对齐,不过似乎没有效果,所以现在采用手动计算对齐的方式。

我尝试过使用\lstinline|mmap| + \lstinline|msync|来达成写入一致性,在完成对\lstinline|mmap|内存的写入后再调用\lstinline|msync|来同步对应的页。经测试这样实现的性能非常差,无论是预先使用\lstinline|ftruncate|来预留文件长度,还是每次写入前先调用\lstinline|ftruncate|来调整文件长度,每次\lstinline|msync|同步一个页都会耗时约7ms \textasciitilde 10ms,与现在的实现有数量级的差异。

\section{测试结果}

\subsection{正确性测试}

提供的所有测试程序都能正确通过,对于\lstinline|Range|操作我也自己编写了一些测试(在\lstinline|test/range_test.cpp|中),也能顺利通过。一组输出如下:

\begin{lstlisting}
======================= single thread test ============================
open engine_path: ./data/test-20274702559820
======================= single thread test pass :) ======================
--------------------------------------
======================= multi thread test ============================
open engine_path: ./data/test-20279021176544
======================= multi thread test pass :) ======================
--------------------------------------
======================= crash test ============================
open engine_path: ./data/test-20710228443442
======================= crash test pass :) ======================
--------------------------------------
======================= range test ============================
open engine_path: ./data/test-20713103413864
======================= range test pass :) ======================
\end{lstlisting}

\subsection{性能测试}

我在自己的笔记本(\lstinline|Intel(R) Core(TM) i5-6300U|)上使用不同参数进行了测试,Throughput(op/s)结果如下。因为这个笔记本的性能相当低下,所以如果使用其它平台来测试的话,结果也许会有显著的提升。

\begin{table}[!htbp]
\begin{center}
  \begin{tabular}{|c|c|c|c|c|c|c|c|c|} \hline
    \diagbox{线程数}{读比例,分布} & 99, 0 & 99, 1 & 90, 0 & 90, 1 & 70, 0 & 70, 1 & 50, 0 & 50, 1\\ \hline
    1 & 332227 & 376357 & 76308 & 79737 & 27431 & 25282 & 13962 & 15567  \\ \hline
    2 & 556985 & 576568 & 78242 & 81228 & 27784 & 20520 & 11431 & 13936 \\ \hline
    3 & 582371 & 659195 & 82718 & 80461 & 21689 & 18849 & 10387 & 13631 \\ \hline
    4 & 463965 & 510715 & 59205 & 49560 & 14246 & 15430 & 10666 & 8828 \\ \hline
    5 & 527341 & 508689 & 56476 & 46661 & 14674 & 13257 & 10121 & 10275 \\ \hline
    6 & 505485 & 552604 & 56899 & 56646 & 13838 & 18941 & 9364 & 9227 \\ \hline
  \end{tabular}
\end{center}
\end{table}

观察第一列可以看出,即使不涉及任何线程冲突,读比例的减低也将显著降低性能,这是因为写操作的开销远远大于读操作,前者只需要操作内存中的数据结构,而后者需要进行硬盘操作。

在线程数不多时,随着线程数的增多,在读比例相对高的情景下性能提升更多,这有可能是因为线程冲突和写锁的开销导致的,也有可能是机器的硬盘的写性能已经达到瓶颈导致的。随着线程数的继续增加,性能会大致稳定到一个值,可以看出读比例越低,稳定时的线程数也越少(然而具体在线程数为多少时稳定,显然受CPU和硬盘性能的影响,这个数据仅供参考)。

此外,当数据分布为1的时候性能有一定的提升(虽然不大),这说明输入数据的分布也会影响到存储引擎的性能。在读比例较高的时候提升较大,这可能是因为在我的视线中,输入数据的分布对内存的访问模式的影响较大,而对硬盘的访问模式的影响较小。

\section{思考问题}

如何保证和验证Key Value存储引擎的Crash Consistency?考虑如下Crash情况:

\begin{enumerate}
  \item KV 崩溃(进程崩溃)
  
  为了测试这种情况,可以使用杀死进程等纯软件手段来使进程崩溃,然后测试数据正确性。
  
  为了在这种情况下保证Crash Consistency,只需要使用正常的\lstinline|write|,或者写\lstinline|mmap|内存,因为即使进程崩溃时内核中有脏的缓存,操作系统依旧会完成后续的处理。注意\lstinline|libc|提供的\lstinline|fwrite|仍然不能满足要求,因为\lstinline|libc|的缓存在用户内存,操作系统不一定会完成后续的处理。
  \item 操作系统崩溃
  
  为了测试这种情况,可以使用虚拟机,如QEMU等模拟断电或者操作系统崩溃,然后测试数据正确性。
  
  为了在这种情况下保证Crash Consistency,可以使用Direct IO,或者\lstinline|write| + \lstinline|fsync|,或者写\lstinline|mmap|内存+\lstinline|msync|。总之是需要让操作系统发出写硬盘的指令并成功完成后,\lstinline|Write|才能返回。
  \item 机器掉电
  
  为了测试这种情况,可以拿真机来运行并断电,然后测试数据正确性。但是这样实际上很难精确控制断电的时机,可能需要相当数量的测试。
  
  为了在这种情况下保证Crash Consistency,普通的操作系统提供的软件接口不太可能能够满足要求,因为即使硬盘控制器返回了成功写入的信息,也有可能只是写入了硬盘的缓存,而没有真正写入持久存储。一些可能(部分)满足要求的方法包括:使用特定设备的特殊的,保证返回成功信息时即保证成功写入持久存储的指令;或者采用校验位之类的策略,排除没有成功写入的数据(这不总能保证\lstinline|Write|返回时写入必定成功,但是至少可以避免错误的数据引发更大的错误)。
\end{enumerate}

\end{document}
