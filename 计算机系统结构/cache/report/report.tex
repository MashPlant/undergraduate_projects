\documentclass[12pt, UTF8]{article}
\usepackage[a4paper, scale = 0.8]{geometry}
\usepackage{ctex}

\usepackage{listings}
\usepackage{xcolor}
\usepackage{color}
\usepackage{multirow}
\usepackage{diagbox}
\definecolor{GrayCodeBlock}{RGB}{241, 241, 241}
\definecolor{BlackText}{RGB}{0, 0, 0}
\definecolor{RedTypename}{RGB}{182, 86, 17}
\definecolor{GreenString}{RGB}{96, 172, 57}
\definecolor{PurpleKeyword}{RGB}{184, 84, 212}
\definecolor{GrayComment}{RGB}{170, 170, 170}
\definecolor{GoldDocumentation}{RGB}{180, 165, 45}
\lstset {
  columns = fullflexible, keepspaces = true, showstringspaces=false, breaklines = true, frame = single, framesep = 0pt, framerule = 0pt, framexleftmargin = 4pt, framexrightmargin = 4pt, framextopmargin = 5pt, framexbottommargin = 3pt, xleftmargin = 4pt, xrightmargin = 4pt,
  backgroundcolor = \color{GrayCodeBlock},
  basicstyle = \ttfamily\color{BlackText},
  keywordstyle = \color{PurpleKeyword},
  ndkeywordstyle = \color{RedTypename},
  stringstyle = \color{GreenString},
  commentstyle = \color{GrayComment}
}

\usepackage{graphicx}
\usepackage{amsmath}

\usepackage[colorlinks, linkcolor = red, anchorcolor = blue, citecolor = green]{hyperref}

\renewcommand\thesection{\arabic{section}}

\title{报告}
\author{李晨昊 2017011466}
\begin{document}
\maketitle
\tableofcontents

\section{运行方法}

直接执行\lstinline|make|即可生成可执行文件\lstinline|main|。直接运行\lstinline|main|,对\lstinline|trace|文件夹中的所有文件执行题目中要求的各种Cache参数组合的测试,包括输出要求的log文件(trace文件名是写死在代码中的,不会考虑\lstinline|trace|文件夹中的实际内容)。因为Cache组内查找和部分替换算法的运行时间都正比于关联路数,所以全相联的情形下总的运算量很大,需要等待较长时间才会得到输出。

\section{Cache布局性能分析}

我在空间开销上已经几乎做到了软件实现的最优:对于缓存行的数组和替换算法的信息的数组,每个逻辑上的元素都是数目相同的\lstinline|u8|单元,其数目是通过计算每个元素所需要的位数除以\lstinline|u8|的大小,向上取整得到的。如果还要进一步优化,可以允许元素不使用完整的\lstinline|u8|单元,不过这样代码的复杂程度也会进一步上升。

\subsection{缺失率}

测试数据如下:

\begin{table}[htbp]
\begin{center}
\begin{tabular}{|c|c|c|c|c|c|c|c|c|c|l|l|l|}
\hline
\multirow{2}*{\diagbox[innerwidth=3cm]{文件}{Cache布局}}   & \multicolumn{3}{c|}{直接映射} & \multicolumn{3}{c|}{4-way组相联} & \multicolumn{3}{c|}{8-way组相联} & \multicolumn{3}{c|}{全相联} \\ \cline{2-13} 
                    & 8       & 32     & 64     & 8         & 32      & 64      & 8         & 32      & 64      & 8       & 32     & 64    \\ \hline
astar         & 23.40   & 9.84   & 5.27   & 23.28     & 9.63    & 5.01    & 23.28     & 9.63    & 5.00    & 23.26   & 9.59   & 4.97  \\ \hline
bodytrack\_1m & 2.06    & 1.33   & 1.59   & 1.22      & 0.31    & 0.15    & 1.22      & 0.31    & 0.15    & 1.22    & 0.31   & 0.15  \\ \hline
bzip2         & 4.24    & 1.34   & 0.85   & 4.11      & 1.20    & 0.68    & 4.10      & 1.19    & 0.68    & 4.09    & 1.19   & 0.67  \\ \hline
canneal.uniq  & 3.67    & 2.31   & 1.89   & 2.07      & 1.14    & 0.85    & 1.79      & 0.82    & 0.62    & 1.75    & 0.66   & 0.39  \\ \hline
gcc           & 6.58    & 2.16   & 1.22   & 6.54      & 2.12    & 1.15    & 6.54      & 2.12    & 1.15    & 6.54    & 2.11   & 1.15  \\ \hline
mcf           & 0.25    & 0.25   & 0.25   & 0.25      & 0.25    & 0.25    & 0.25      & 0.25    & 0.25    & 0.25    & 0.25   & 0.25  \\ \hline
perlbench     & 1.07    & 1.07   & 1.07   & 0.76      & 0.76    & 0.76    & 0.76      & 0.76    & 0.76    & 0.75    & 0.75   & 0.76  \\ \hline
streamcluster & 4.94    & 2.20   & 1.46   & 4.58      & 1.82    & 1.08    & 4.58      & 1.82    & 1.08    & 4.58    & 1.82   & 1.08  \\ \hline
swim          & 4.34    & 4.34   & 4.34   & 2.65      & 2.65    & 2.65    & 2.38      & 2.38    & 2.38    & 0.85    & 1.28   & 2.05  \\ \hline
twolf         & 1.18    & 0.39   & 0.27   & 1.14      & 0.34    & 0.20    & 1.14      & 0.34    & 0.20    & 1.14    & 0.34   & 0.20  \\ \hline
\end{tabular}
\end{center}
\end{table}

观察数据可以看出,随着相联度的上升,缺失率有一定下降,但是不是很明显;随着块大小的上升,缺失率一般会明显下降。当然,缺失率不会一直随着块大小的上升而下降,我尝试了一下更大的块大小,缺失率会先下降再上升,只是要求测试的区间全部包含在了缺失率下降的区间中。

不同的文件对缺失率也有明显的影响,对于某些文件,Cache布局对缺失率的影响很小。

\subsection{空间开销}

Cache的布局会影响缓存行大小,缓存行个数;替换算法所需的信息单元大小,信息单元个数。不过因为实现的限制,每个逻辑上的数组元素都是由一定数量的\lstinline|u8|单元保存的,所以数组元素的位大小变化时,不一定能在实际申请的内存中体现出来。

具体来说,这些值的计算方式分别为:
\begin{align*}
\text{缓存行大小(只考虑元信息)} &= 64 - (\log_2\text{缓存容量} - \log_2\text{关联路数}) + 1 + [\text{写回}] \\
\text{缓存行个数} &= \text{缓存容量} / \text{块大小} \\
\text{信息单元大小(LRU)} &= \text{关联路数} \times \log_2\text{关联路数} \\
\text{信息单元个数} &= \text{缓存容量} / \text{块大小} / \text{关联路数}
\end{align*}

值得注意的是,缓存行大小不受块大小的影响。这是因为块大小会同时影响地址中的索引和块内偏移两个部分,二者的影响抵消了。

容易看出,单个块越大,关联度越低,缓存行总空间和信息单元总空间越小。

\section{替换算法性能分析}

除了要求的三个算法之外,我还实现了LFU替换算法。我对原始的算法进行了一点改动:当一行的计数值达到关联度数时,把同一组内的所有计数值都除以2。这样可以避免原始算法的一个问题:历史上访问次数很多而最近没有被访问的行可能迟迟不被换出,同时控制了计数值的最大值($\text{关联路数} - 1$),可以减少算法的空间消耗。

\subsection{算法实现}

维护信息(命中一行时):

\begin{enumerate}
  \item LRU:对于一组内的每个索引值,如果它比命中的行的索引值小则加一,相等则置零,否则不变。这就是模拟把这一行从栈中抽出而移到栈顶,在它上方的行都向下移动一个位置,在它下方的行都不动。
  
\begin{lstlisting}[language = C++, morekeywords = { u32 }, ndkeywords = { read_bits, write_bits }]
// read_bits和write_bits分别实现从指针p偏移start个bit处读出/写入长度为len的比特序列,这个实现要求len < 16
// _bextr_u32是BMI指令集提供的intrinsic,任何常见的CPU都应该支持它。 _bextr_u32(x, start, len)读出32位整数的比特区间[start, start + len)的值
u32 read_bits(const u8 *p, u32 start, u32 len) {
  p += (start >> 4) << 1, start &= 15;
  return _bextr_u32(*(u32 *) p, start, len);
}
void write_bits(u8 *p, u32 start, u32 len, u32 val) {
  p += (start >> 4) << 1, start &= 15;
  *p1 = *p1 ^ _bextr_u32(*p1, start, len) << start | val << start;
}
...
u32 cur = read_bits(p, i * lg_nway, lg_nway); // 读出命中的行的索引值
for (u32 j = 0; j < nway; ++j) {
  // 读出某行的索引值,计算新的索引值并存入
  u32 val = read_bits(p, j * lg_nway, lg_nway);           
  u32 new_ = val < cur ? val + 1 : val == cur ? 0 : val;
  write_bits(p, j * lg_nway, lg_nway, new_);
}
\end{lstlisting}

  \item TREE:在一棵数组表示的完美二叉树上,从根走到命中的行对应的叶子节点,每个中间节点的值都置为表示指向另一侧的值。

\begin{lstlisting}[language = C++, morekeywords = { u32 }, ndkeywords = { access_tree, read_bit, write_bit }]
// read_bit和write_bit是常规的bitset的实现,没有什么必要列出来。从功能上来说它们可以用read_bits和write_bits来实现,不过单独实现它们性能更好一些
void access_tree(u8 *p, u32 way) {
  u32 idx = 0;
  // ~j >> 31和有符号整数的j >= 0语义相同,只是我基本一直都在用无符号整数,所以就不想改成有符号整数了
  for (u32 j = lg_nway - 1; ~j >> 31; --j) {
    bool right = way >> j & 1; // right表示是否往右走
    write_bit(p, idx, !right); // 若往右走,"告诉别人往左走"
    idx = (idx << 1) + 1 + right; // 计算往右走的下标
  }
}
...
access_tree(p, i); // 在树上标记刚才访问的行
\end{lstlisting}
  \item RAND:无操作。
  \item LFU:将命中的行的计数值加一,若加一后的值等于关联度数,则将一行内所有计数值都折半。代码比较简单,这里就不分析了。
\end{enumerate}

替换动作:

\begin{enumerate}
  \item LRU:选择索引值等于$\text{关联路数} - 1$的行替换掉。同时将所有的索引值都加1(相当于整体向下移动一个单位),将替换掉的这一行的索引值重新置为0(表示它是刚刚访问的)。

\begin{lstlisting}[language = C++, morekeywords = { u32 }, ndkeywords = { read_bits, write_bits }]
// fill将一行标记为valid
// 根据计算,一行的位数(只考虑元数据)一定<= 64,所以可以直接操作u64
void fill(u32 line_base, u32 way, u64 tag) {
  u64 *pline = (u64 *) &lines[line_base + way * linesz];
  // 这样只会修改标记为和tag,其余位保持不变(因为这个u64并不都属于这行)
  *pline = 1 | tag << 1 | *pline ^ _bextr_u64(*pline, 0, linesz * 8);
}
...
for (u32 i = 0; i < nway; ++i) {
  u32 val = read_bits(p, i * lg_nway, lg_nway);
  u32 new_ = val + 1;
  // free_way是前面遍历中找到的未填的行,若不存在则是-1
  if (val == nway - 1 || i == free_way) {
    new_ = 0, fill(line_base, i, tag);
  }
  write_bits(p, i * lg_nway, lg_nway, new_);
}
\end{lstlisting}
  \item TREE:在一颗数组表示的完美二叉树上,从树根按照每个节点的值的指向,依次走到一个叶子节点,把它替换掉。同时使用与维护信息时一样的算法,从根节点走到这个叶子并标记沿途的节点。

\begin{lstlisting}[language = C++, morekeywords = { u32 }, ndkeywords = { access_tree, read_bit, write_bit }]
if (free_way == -1u) {
  free_way = 0;
  for (u32 i = lg_nway - 1; ~i >> 31; --i) {
    // 按照节点的指示走到一个叶子节点
    free_way = (free_way << 1) + 1 + read_bit(p, free_way);
  }
  free_way -= nway - 1; // 需要注意叶子节点的下标并不直接对应行的索引,差一个偏移量
}
access_tree(p, free_way);
fill(line_base, free_way, tag);
\end{lstlisting}
  \item RAND:随机选择一行替换掉。
  \item LFU:选择计数值最小的行替换掉,同时把这一行的计数值重新置为1(表示它最近访问过一次)。代码比较简单,这里就不分析了。
\end{enumerate}

\subsection{缺失率}

测试数据如下:

\begin{table}[htbp]
\begin{center}
  \begin{tabular}{|l|l|l|l|l|}
    \hline
    \diagbox[innerwidth=3cm]{文件}{替换算法} & LRU & TREE  & RAND  & LFU   \\ \hline
    astar         & \multicolumn{1}{c|}{23.28} & 23.29 & 23.22 & 23.23 \\ \hline
    bodytrack\_1m & 1.22                       & 1.22  & 1.22  & 1.25  \\ \hline
    bzip2         & 4.10                       & 4.09  & 4.12  & 4.12  \\ \hline
    canneal.uniq  & 1.79                       & 1.78  & 1.79  & 2.53  \\ \hline
    gcc           & 6.54                       & 6.54  & 6.58  & 6.54  \\ \hline
    mcf           & 0.25                       & 0.25  & 0.25  & 0.25  \\ \hline
    perlbench     & 0.76                       & 0.77  & 0.83  & 0.86  \\ \hline
    streamcluster & 4.58                       & 4.58  & 4.60  & 4.58  \\ \hline
    swim          & 2.38                       & 2.48  & 3.01  & 2.76  \\ \hline
    twolf         & 1.14                       & 1.14  & 1.14  & 1.14  \\ \hline
  \end{tabular}
\end{center}
\end{table}

从数据可以看出,在这个给定的Cache布局下,LRU算法的表现总体来看最好,TREE算法那比它差一些,但是也很接近,这是符合预期的。而我实现LFU的缺失率表现比较一般。不过我还测试了块大小更大的情形,LFU的相对表现会变好很多,甚至在很多测例中达到最好。

\subsection{空间开销}

\begin{itemize}
  \item LRU:每组需要$\text{关联路数} \times \log_2\text{关联路数}$位的信息。
  \item TREE:每组需要$\text{关联路数} - 1$位的信息。
  \item RAND:代码层面上不保存任何信息,看起来是零空间开销的。如果实际在硬件上实现,也可以只用常数空间维护一个随机数种子。
  \item LFU:每组需要$\text{关联路数} \times \log_2\text{关联路数}$位的信息,与LRU算法一致。
\end{itemize}

\section{写策略性能分析}

\subsection{缺失率}

如果Cache替换算法没有考虑写回引入的dirty位,那么写策略选择写回还是写直达对Cache的缺失率没有任何影响。因此这里只考虑写分配与写不分配。

测试数据如下:

\begin{table}[htbp]
  \begin{center}
    \begin{tabular}{|l|l|l|}
      \hline
      \diagbox[innerwidth=3cm]{文件}{写策略} & 写分配 & 写不分配 \\ \hline
      astar         & 23.28 & 34.50   \\ \hline
      bodytrack\_1m & 1.22 &  8.67  \\ \hline
      bzip2         & 4.10 &  8.67  \\ \hline
      canneal.uniq  & 1.79 &  4.66  \\ \hline
      gcc           & 6.54 &  9.61  \\ \hline
      mcf           & 0.25 &  0.25  \\ \hline
      perlbench     & 0.76 &  0.76  \\ \hline
      streamcluster & 4.58 &  11.15  \\ \hline
      swim          & 2.38 &  2.38  \\ \hline
      twolf         & 1.14 &  1.45  \\ \hline
    \end{tabular}
  \end{center}
\end{table}     

可以看出,除了少数trace,写分配都的缺失率明显优于写不分配的确实率,这是符合预期的。

\subsection{空间开销}

写策略选择写分配还是写不分配对空间开销没有影响。写策略选择写回时,每个缓存行的元数据需要多一个dirty位来记录这个缓存行是否被写过,用于后续换出时决定是否需要写内存。

\end{document}