\documentclass[12pt, UTF8]{article}
\usepackage[a4paper, scale = 0.8]{geometry}
\usepackage{ctex}

\usepackage{enumitem}
\usepackage{listings}
\usepackage{xcolor}
\usepackage{color}
\definecolor{GrayCodeBlock}{RGB}{241, 241, 241}
%\definecolor{BlackText}{RGB}{110, 107, 94}
\definecolor{BlackText}{RGB}{0, 0, 0}
\definecolor{RedTypename}{RGB}{182, 86, 17}
\definecolor{GreenString}{RGB}{96, 172, 57}
\definecolor{PurpleKeyword}{RGB}{184, 84, 212}
\definecolor{GrayComment}{RGB}{170, 170, 170}
\definecolor{GoldDocumentation}{RGB}{180, 165, 45}
\lstset {
  columns = fullflexible, keepspaces = true, showstringspaces=false, breaklines = true, frame = single, framesep = 0pt, framerule = 0pt, framexleftmargin = 4pt, framexrightmargin = 4pt, framextopmargin = 5pt, framexbottommargin = 3pt, xleftmargin = 4pt, xrightmargin = 4pt,
  backgroundcolor = \color{GrayCodeBlock},
  basicstyle = \ttfamily\color{BlackText},
  keywordstyle = \color{PurpleKeyword},
  ndkeywordstyle = \color{RedTypename},
  stringstyle = \color{GreenString},
  commentstyle = \color{GrayComment}
}

\usepackage{graphicx}
\usepackage{amsmath}

\usepackage[colorlinks, linkcolor = red, anchorcolor = blue, citecolor = green]{hyperref}

\renewcommand\thesection{\arabic{section}}

\title{网络编程技术TCP编程作业报告}
\author{李晨昊 2017011466}
\begin{document}
\maketitle
\tableofcontents

\section{运行方式}

直接执行\lstinline|make|即可编译出\lstinline|server|(需要支持\lstinline|C++ 17|的编译器)。\lstinline|server|可以带一个参数执行或者直接执行,参数用于指示监听的地址,不带参数时默认监听0.0.0.0地址。总是监听8000端口。

我自己已经编写了一个客户端软件来验证它的正确性,只是没有一起提交上来。

\section{协议设计}

我实现了一个简单的带登录功能的文件上传/下载服务器。

用户建立连接后,首先需要依次发送两个大端序16位无符号整数,分别表示接下来要发送的用户名和密码的长度。接着依次发送对应长度的用户名和密码。发送的这些数据之间都没有分隔符,后续的操作也是一样的。服务器将校验用户名和密码,如果用户名已经存在则视为登录操作,取出文件中的密码与之进行比对;否则视为注册操作。

完成登陆/注册后,对于每个操作,用户先发送两个大端序32位无符号整数,称之为\lstinline|arg1|和\lstinline|arg2|。\lstinline|arg1|的低两位表示操作类型,总共有四种:

\begin{enumerate}[label = \arabic*:, start = 0]
  \item 下载文件:arg1的17-3位表示文件名长度。接着发送对应长度的文件名,服务器完成文件读取后发送文件内容。
  \item 上传文件:arg1的17-3位表示文件名长度,arg2表示文件长度。接着依次发送对应长度的文件名和文件,服务器保存这个文件,接着发送信息表示成功。
  \item 列出所有文件:服务器发送一个字符串,其内容与\lstinline|ls|的结果类似。
  \item 中断连接。
\end{enumerate}

在以上操作的任何一步中遇到错误时,服务器都会发送一个错误信息并且中断连接。唯一的一个例外是如果试图下载一个不存在的文件,服务器只会发送一个错误信息,而不会中断连接。

所有服务器发送的信息的格式都是先发送两个大端序32位无符号整数,第一个整数为零表示操作成功,否则为失败。第二个整数表示接下来的信息的长度,然后发送对应长度的字节串。

\section{程序架构设计}

\subsection{基本设计}

我实现的是一个简单的基于\lstinline|pthread|的多线程服务器,主线程不断监听到来的连接,建立好连接之后派生出线程,每个线程负责一个连接的处理。处理连接的时候,首先进行登录/注册操作,为每个用户建立一个密码文件和一个用于存储用户文件的文件夹,然后开始类似REPL的收发过程。

我考虑了一种可能的线程冲突的情景,即多个线程同时执行注册操作时可能会发生不一致,所以我在这里使用了锁。其余地方我都没有使用锁,如果是在服务过程中发生的冲突,可能对用户而言会产生困惑,但是不会影响到服务器的工作状态,所以无需保护。

我为socket设定了一些课上提到的参数,包括\lstinline|SO_REUSEADDR|,\lstinline|TCP_NODELAY|,\lstinline|SO_LINGER|等,还设置了忽略\lstinline|SIGPIPE|信号,detach派生出的线程。

\subsection{资源管理}

\lstinline|C++|中提倡用RAII-style来管理和释放资源,但是我们往往需要和文件描述符,\lstinline|mmap|的内存等打交道,这些在标准库中并没有直接的管理者,如果都要用RAII-style的类管理起来的话会增加很多代码量,所以我选择了一种折衷的办法,仍然需要程序员手动指定释放资源的操作,但是不需要指定释放资源的时机,而是在当前代码块结束后自动释放:

\begin{lstlisting}[language = C++, ndkeywords = { DEFER }]
// need C++17 template deduction guide
#define DEFER(f) Defer DEFER_0(__deferred)([&]() { f; })

template <typename F>
struct Defer {
  F f;
  Defer(F f) : f(f) {}
  ~Defer() { f(); }
};
\end{lstlisting}

这是模仿\lstinline|Go|的资源管理管理方式,不过实现的机制和\lstinline|Go|并不一样。在一个语句块中使用\lstinline|DEFER(f)|,就会在这个地方定义一个保存了这个操作的变量,在它析构时会执行这个操作。这里\lstinline|DEFER_0|宏是用来给变量名字加上一个计数器后缀的,这样可以允许一个块中多次使用\lstinline|DEFER|。

\subsection{错误处理}

在我的错误处理的设计中,最核心的部分是\lstinline|TRY|宏:

\begin{lstlisting}[language = C++, ndkeywords = { TRY }]
#define TRY(op, require) \
  ({ auto __x = op; if (!(__x require)) return perror(#op), false; __x; })
\end{lstlisting}

这就相当于\lstinline|Rust|中的\lstinline|try!|宏或者是\lstinline|?|运算符的一个非常简易的模拟。它的作用是在一个返回\lstinline|bool|值的函数中执行一个操作,如果返回值不满足要求则输出错误信息(这不是必要的,仅做调试用途)并返回\lstinline|false|,中止当前的执行并且把错误信息报告给上层;如果满足则将这个返回值作为一个表达式参与后续操作。一个简单的例子如下,这是我的上传文件的实现的一部分:

\begin{lstlisting}[language = C++, morekeywords = { u32, nullptr }, ndkeywords = { TRY }]
if (![=] {
  Buffer p = get_path(connfd, arg1 >> 2 & 0xFFFF);
  int fd = TRY(openat(dirfd, p.get(), O_RDWR | O_CREAT | O_TRUNC, 0666), > 0); // 打开文件,只有返回值 > 0时才继续操作
  DEFER(close(fd));
  TRY(ftruncate(fd, arg2), == 0); // 调整文件大小,检测返回值为0后丢弃之
  ...
  return true;
}()) { ERRMSG("failed to put"); } // 有一个操作失败了,向用户发送错误信息
\end{lstlisting}

\end{document}